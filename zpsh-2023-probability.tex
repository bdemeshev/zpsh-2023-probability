% arara: xelatex: {shell: yes}
%% arara: biber
%% arara: xelatex: {shell: yes}
%% arara: xelatex: {shell: yes}
\documentclass[12pt]{article}

\usepackage{etoolbox} % для условия if-else
\newtoggle{excerpt} % помечаем, что это отрывок, а далее в тексте может использовать
\toggletrue{excerpt}
% команду \iftoggle{excerpt}{yes}{no}

\usepackage{tikz} % картинки в tikz
\usepackage{microtype} % свешивание пунктуации

\usepackage{array} % для столбцов фиксированной ширины

\usepackage{indentfirst} % отступ в первом параграфе

\usepackage{sectsty} % для центрирования названий частей
\allsectionsfont{\centering}

\usepackage{amsmath} % куча стандартных математических плюшек

\usepackage{comment}
\usepackage{amsfonts}

\usepackage[top=2cm, left=1.2cm, right=1.2cm, bottom=2cm]{geometry} % размер текста на странице

\usepackage{lastpage} % чтобы узнать номер последней страницы

\usepackage{enumitem} % дополнительные плюшки для списков
%  например \begin{enumerate}[resume] позволяет продолжить нумерацию в новом списке
\usepackage{caption}


\usepackage{fancyhdr} % весёлые колонтитулы
\pagestyle{fancy}
\lhead{ЗПШ-2023}
\chead{}
\rhead{Невероятные задачи по вероятностям}
\lfoot{}
\cfoot{Всем ежа!}
\rfoot{}
\renewcommand{\headrulewidth}{0.4pt}
\renewcommand{\footrulewidth}{0.4pt}



\usepackage{todonotes} % для вставки в документ заметок о том, что осталось сделать
% \todo{Здесь надо коэффициенты исправить}
% \missingfigure{Здесь будет Последний день Помпеи}
% \listoftodos --- печатает все поставленные \todo'шки


% более красивые таблицы
\usepackage{booktabs}
% заповеди из докупентации:
% 1. Не используйте вертикальные линни
% 2. Не используйте двойные линии
% 3. Единицы измерения - в шапку таблицы
% 4. Не сокращайте .1 вместо 0.1
% 5. Повторяющееся значение повторяйте, а не говорите "то же"

\usepackage{multicol}


\usepackage{fontspec}
\usepackage{polyglossia}

\usepackage{tikzlings}
\usepackage{tikzducks}

\setmainlanguage{russian}
\setotherlanguages{english}

% download "Linux Libertine" fonts:
% http://www.linuxlibertine.org/index.php?id=91&L=1
\setmainfont{Linux Libertine O} % or Helvetica, Arial, Cambria
% why do we need \newfontfamily:
% http://tex.stackexchange.com/questions/91507/
\newfontfamily{\cyrillicfonttt}{Linux Libertine O}

\AddEnumerateCounter{\asbuk}{\russian@alph}{щ} % для списков с русскими буквами
\setlist[enumerate, 2]{label=\asbuk*),ref=\asbuk*} % \alph* \asbuk* \arabic*

%% эконометрические сокращения
\DeclareMathOperator{\Cov}{Cov}
\DeclareMathOperator{\Corr}{Corr}
\DeclareMathOperator{\Var}{Var}
\DeclareMathOperator{\E}{E}
\newcommand \hb{\hat{\beta}}
\newcommand \hs{\hat{\sigma}}
\newcommand \htheta{\hat{\theta}}
\newcommand \s{\sigma}
\newcommand \hy{\hat{y}}
\newcommand \hY{\hat{Y}}
\newcommand \e{\varepsilon}
\newcommand \he{\hat{\e}}
\newcommand \z{z}
\newcommand \hVar{\widehat{\Var}}
\newcommand \hCorr{\widehat{\Corr}}
\newcommand \hCov{\widehat{\Cov}}
\newcommand \cN{\mathcal{N}}
\let\P\relax
\DeclareMathOperator{\P}{\mathbb{P}}

\DeclareMathOperator{\plim}{plim}



\usepackage{color,url,amsthm,amssymb,longtable,eurosym}

% делаем короче интервал в списках
\setlength{\itemsep}{0pt}
\setlength{\parskip}{0pt}
\setlength{\parsep}{0pt}

\begin{document}


\section{Выйду ночью в плоскость с ежом!}

\begin{enumerate}


\item Аня хватается за верёвку в форме окружности в произвольной точке.
    Боря берёт мачете и с завязанными глазами разрубает верёвку в двух случайных независимых местах. 
    Аня забирает себе тот кусок, за который держится. Боря забирает оставшийся кусок. 
    Вся верёвка имеет единичную длину.
%      \item Чему равна ожидаемая длина куска верёвки, доставшегося Ане?
      
Чему равна вероятность того, что у Ани верёвка длиннее?

\item Аня и Боря приходят в столовую независимо друг от друга от 9:30 до 9:50 в случайный момент времени и завтракают за 5 минут. 

Какова вероятность того, что они увидятся?


\item Города Левск и Правск соединены железной дорогой. 
Поезда в обе стороны отправляются из них каждый час одновременно, время в пути составляет ровно час. 
Стрелочник, живущий в домике при железной дороге, любит подойти к окну в случайный момент времени, 
дождаться первого проходящего мимо поезда и записать его направление. 
Поезда обоих направлений в его записях встречаются одинаково часто. 
  
  \begin{enumerate}
      \item В скольки минут пути на поезде от ближайшего города он живёт?
      \item Сколько в среднем он ждёт поезда?
  \end{enumerate}


\item 

На планету Плюк, окружность, в случайных точках садятся $3$ пепелаца.
Радиосвязь между двумя точками на планете Плюк возможна,
если центральный угол между этими двумя точками меньше $\pi/2$.

\begin{enumerate}
\item Какова вероятность того,
что из любой точки планеты можно связаться хотя бы с одним пепелацем?
\item Какова вероятность того,
все три пепелаца смогут поддерживать связь друг с другом
(необязательно напрямую, возможно через посредника)?
\end{enumerate}

\item Странник вышел из домика в 9 утра и в 9 вечера достиг вершины Фудзи. 
Переночевав на вершине, в 9 утра следующего дня он начал спускаться вниз по той же дороге и 
в 9 вечера вернулся домой. Шёл странник с разной скоростью и делал остановки в разных местах.

Обязательно ли найдётся место на дороге, в котором он оказался в то же время дня на подъеме и на спуске?


\end{enumerate}

\newpage
\section{Безусловные ежи сажают деревья!}

\begin{enumerate}
    \item Имеется три монетки. Две «правильных» и одна — с «орлами» по обеим сторонам. 
    Петя выбирает одну монетку наугад и подкидывает её два раза. 
    Оба раза выпадает «орел». 

    Какова условная вероятность того, что монетка «неправильная»?    
    
    \item  В городе примерно 4\% такси зелёного цвета и остальные жёлтые. Свидетель путает цвет на показаниях в суде с вероятностью 10\%.

\begin{enumerate}
\item Какова вероятность того, свидетель скажет, что видел зелёное такси?
\item Какова вероятность того, свидетель ошибётся?
\item Какова вероятность того, что такси было зелёным, если свидетель говорит, что оно было зелёным?
\item Какова вероятность того, что такси было жёлтым, если свидетель говорит, что оно было жёлтым?
\end{enumerate}


    \item У тети Маши — двое детей, разного возраста. Вероятности рождения мальчика и девочки равны и не зависят от дня недели, а пол первого и второго ребенка независимы. 
    Найдите вероятность того, что у тёти Маши дети обоих полов, если\ldots
    \begin{enumerate}
    \item \ldots{ } хотя бы один её ребенок — мальчик.
    \item \ldots{ } старший ребенок — мальчик.
    \item \ldots{ } на вопрос: «А правда ли тетя Маша, что у вас есть сын, родившийся в пятницу?» тётя Маша ответила: «Да».
    \item \ldots{ } в лотерее участвовали оба ребёнка, и главный приз достался мальчику.
    \end{enumerate}
    

    \item Ты смертельно болен. Спасти тебя может только один вид  целебной лягушки. Целебны у этого вида только самцы. Самцы и самки встречаются равновероятно. Ты на дороге и предельно ослаб. Слева в 100 метрах от тебя одна лягушка целебного вида, но не ясно, самец или самка. Справа в 100 метров аж две лягушки целебного вида, но тоже издалека неясно кто. От двух лягушек в твою сторону дует ветер, поэтому ты можешь их слышать.

В какую сторону стоит ползти из последних сил в каждой из  ситуаций?
\begin{enumerate}
  \item Cамцы и самки квакают одинаково, со стороны правых двух лягушек ты слышишь кваканье.
  \item Самки квакают, самцы — нет, со стороны правых двух лягушек ты слышишь кваканье, но не разобрать, одной лягушки или двух.
  \item Самцы и самки квакают по разному, но одинаково часто. Ты слышишь отдельный квак одной из двух лягушек справа и это квак самки.
\end{enumerate}

\item 
Есть три закрытых двери. За двумя из них — по козе, за третьей автомобиль. Ты выбираешь одну из дверей. Допустим, ты выбрал дверь А. Ведущий шоу открывает дверь B и за ней нет автомобиля.
В этот момент ведущий предлагает тебе изменить выбор двери.

Имеет ли смысл изменить выбор в каждой из трёх ситуаций?
\begin{enumerate}
  \item Ведущий выбирал одну из трёх дверей равновероятно.
  \item Ведущий выбирал одну из двух дверей не выбранных тобой равновероятно.
  \item Ведущий выбирал дверь без машины и не совпадающую с твоей.
\end{enumerate}


\end{enumerate}

\newpage
\section{Нежданные ежи встречают уравнения!}
\begin{enumerate}
    \item  Неправильную монетку с вероятностью «орла» равной $0.7$ подбрасывают до первого «орла».
    \begin{enumerate}
        \item Какова вероятность чётного числа бросков? 
        \item Чему равно среднее количество подбрасываний?  Среднее количество решек? 
    \end{enumerate}
    
    \item Саша и Маша по очереди подбрасывают кубик до первой шестёрки. 
    Посуду будет мыть тот, кто первым выбросит шестерку. 
    Маша бросает кубик первой.
    
    Какова вероятность того, что посуду будет мыть Маша? 
    
    \item Пират Злопамятный Джо очень любит неразбавленный ром. Из-за того,
    что он много пьёт, у него проблемы с памятью, и он помнит не больше, чем три последних
    пинты. Хозяин таверны «Огненная зебра» с вероятностью $1/8$ разбавляет каждую подаваемую пинту рома.
    Если по ощущением Джо половина выпитых пинт или больше была разбавлена, то он
    разносит таверну к чертям собачьим. Только что Джо вошёл в таверну и закал первую пинту.
    
    Сколько в среднем пинт выпьет Джо, прежде чем разнесёт таверну?
    

    \item Вася нажимает на пульте телевизора кнопку «On-Off» 100 раз
    подряд. Пульт старый, поэтому в первый раз кнопка срабатывает с
    вероятностью $\frac{1}{2}$, затем вероятность срабатывания падает.
    
    
    Какова вероятность того, что после всех нажатий телевизор будет
    включен, если сейчас он выключен?
    
    \item Есть три комнаты. В первой из них лежит сыр. Если мышка
    попадает в первую комнату, то она находит сыр через одну минуту.
    Если мышка попадает во вторую комнату, то она ищет сыр две минуты
    и покидает комнату. Если мышка попадает в третью комнату, то она
    ищет сыр три минуты и покидает комнату. Покинув комнату, мышка
    выходит в коридор и выбирает новую комнату наугад, например, может
    зайти в одну и ту же. Сейчас мышка в коридоре. 
    
    
    Сколько времени в среднем потребуется мышке, чтобы найти сыр?
    
    \item В каждой вершине треугольника по ёжику. Каждую минуту с вероятностью $0.5$ каждый ежик
    независимо от других двигается по часовой стрелке, с вероятностью
    $0.5$ — против часовой стрелки.
    
    Сколько в среднем времени пройдёт до их встречи?
    

    \item Илье Муромцу предстоит дорога к камню. От камня начинаются ещё три дороги.
    Каждая из тех дорог снова оканчивается камнем. И от каждого камня начинаются ещё три дороги.
    И каждые те три дороги оканчиваются камнем\ldots И так далее до бесконечности.
    На каждой дороге живёт трёхголовый Змей Горыныч.
    Каждый Змей Горыныч бодрствует независимо от других с вероятностью (хм, Вы не поверите!) одна третья.
    У Василисы Премудрой существует Чудо-Карта, на которой видно,
    какие Змеи Горынычи бодрствуют, а какие — нет.
    
    \begin{enumerate}
        \item Какова вероятность того, что путь Ильи Муромца будет проходить 
        исключительно мимо спящих Змеев Горынычей, если 
        Илья Муромец выбирает каждый раз дорогу наугад?
        \item Какова вероятность того,
        что Василиса Премудрая сможет найти на карте
        бесконечный жизненный путь Ильи Муромца проходящий исключительно мимо спящих Змеев Горынычей?
    \end{enumerate}
\end{enumerate}

\newpage
\section{Ещё больше ежиных радостей!}
\begin{enumerate}
\item В самолёте $100$ мест и все билеты проданы.
Первой в очереди на посадку стоит Сумасшедшая Старушка,
она очень переживает, что ей не хватит места.
Сумасшедшая Старушка врывается в самолёт
и несмотря на номер по билету садится на случайно выбираемое место.
Каждый оставшийся пассажир садится на своё место, если оно свободно,
и на случайное выбираемое место, если его место уже кем-то занято.

\begin{enumerate}
\item Какова вероятность того, что все пассажиры сядут на свои места?
\item Какова вероятность того, что второй пассажир в очереди сядет на своё место?
\item Какова вероятность того, что последний пассажир сядет на своё место?
\end{enumerate}

\item Два лекарства испытывали на мужчинах и женщинах. Каждый
человек принимал только одно лекарство. Общий процент людей,
почувствовавших улучшение, больше среди принимавших лекарство А.
Процент мужчин, почувствовавших улучшение, больше среди мужчин, принимавших лекарство В.
Процент женщин, почувствовавших улучшение, больше среди женщин, принимавших лекарство В.

Возможно ли это?

\item Ефросинья подкидывает правильную монетку неограниченное количество раз.
\begin{enumerate}
    \item Какова вероятность того, что два орла подряд выпадут раньше трёх решек подряд?
    \item Какова вероятность того, что последовательность ООР появится раньше ОРР? 
\end{enumerate}


\item Злобный Дракон поймал принцесс Настю и Сашу и посадил в разные башни.
Перед каждой из принцесс Злобный Дракон подбрасывает один раз правильную монетку.
А дальше даёт каждой из них шанс угадать, как выпала монетка у её подруги.
Если хотя бы одна из принцесс угадает, то Злобный Дракон отпустит принцесс на волю.
Если обе принцессы ошибутся, то они навсегда останутся у него в заточении.

Подобная практика у Злобного Дракона исследователями была отмечена уже давно,
поэтому принцессы имели достаточно времени договориться на случай вероятного похищения.

Как следует поступать принцессам при подобных похищениях?

\item Андрей Абрикосов, Борис Бананов и Вова Виноградов играют одной командой в игру.
В комнате три занумерованных закрытых коробки: с абрикосами, бананами и виноградом.
Общаться после начала игры они не могут, но могут заранее договориться о стратегии.
Они заходят в комнату по очереди.
Каждый из них может открыть две коробки по своему выбору, перемещать коробки нельзя.
Перед следующим игроком коробки закрываются. Если Андрей откроет коробку абрикосами,
Борис — с бананами, а Вова — с виноградом, то их команда выигрывает.
Если хотя бы один из игроков не найдёт свой фрукт, то их команда проигрывает.

Какова вероятность выигрыша при использовании оптимальной стратегии?

\end{enumerate}

\newpage
\section{Заметки}

Первое занятие. Пять человек: две семиклассницы, одна десятиклассница, два студента. 
Разобрали 1.1, 1.2, прочитали условие 1.3. 
Петя указал, что надо как-то решать случай ничьей.

Второе занятие. Шесть человек: две семиклассницы, две десятиклассницы, один одиннадцатиклассник, одна одиннадцатиклассница,
один студент. 
Дорешали задачу 1.3 про стрелочника. 
Решили 2.1 и 2.3 пункты а и в. 
Вообще не вводил формулу условной вероятности. 
Вместо неё рисовали дерево и заполняли табличку с результатами при большом количестве опытов. 

Третье занятие. Шесть человек: две семиклассницы, две десятиклассницы, одна одиннадцатиклассница,
один студент. Доразбирали 2.3 пункт г и на новую тему 3.1.
Лада (?) заметила, что у 2.3 пункта г есть две вариации, в зависимости от того, 
может ли достаться приз не ребёнку тёти Маши. 

Четвертое занятие. Семь человек: две семиклассницы, две десятиклассницы, один одиннадцатиклассник, одна одиннадцатиклассница,
один студент. Разобрали пару задач с лабиринта: 
\begin{enumerate}
    \item  Ровно половина жителей острова Невезения зайцы, а остальные — кролики. 
    Зайцы врут в половине своих фраз, а кролики — в двух третях. 
    Вышел однажды житель острова, сел на пенёк и сказал: «Я не кролик». 
    Какова условная вероятность того, что он действительно не кролик?
    \item В пакетике 6 оранжевых и $n$ жёлтых конфет. Аня достаёт одну наугад и съедает. 
    Затем достаёт ещё одну и снова съедает. 
    Вероятность того, что Аня съела две оранжевых равна $1/3$. Сколько конфет было в пакетике?
\end{enumerate}
Затем разобрали 3.5, 3.6 и начали 3.3 про Джо. 
Два математических ожидания в задаче 3.6 про ежей обозначили буквами ё и ж. 

Пятое занятие. Семь человек: две семиклассницы, две десятиклассницы, один одиннадцатиклассник, одна одиннадцатиклассница,
один студент.
Разобрали задачу 3.3 про Джо, затем 4.3 про монетки. 
Дальше я рассказал, что в «дуэли на монетках» преимущество у того дуэлянта, который выбирает 
последовательность вторым и нарисовал граф оптимальных ответов. 

Для конференции школьники сами выбрали и презентовали задачи 1.3 про стрелочника, про зайцев и кроликов с острова невезения и 4.3 про ООР или ОРР.
    


\end{document}
